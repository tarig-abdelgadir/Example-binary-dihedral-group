\documentclass[11pt]{amsart}

%%%Titles and Authors%%%%%%%%%%%%%%%%%%%%%%
\title{Example: binary dihedral group}

\author{Tarig Abdelgadir}
\address{The Abdus Salam International Centre for Theoretical Physics, 
Stada Costiera 11, 
Trieste 34151, 
Italy
}
\email{tabdelga@ictp.it}

\author{Daniel Chan}
\address{School of Mathematics and Statistics, 
UNSW Sydney, 
NSW 2052,	
Australia
}
\email{danielc@unsw.edu.au}

%-------------------------Packages-------------------------
\usepackage{mymacros,amssymb}
\usepackage{fullpage,tikz}
\usepackage{tikz-cd}
\usepackage{pst-node}
\usetikzlibrary{graphs,angles,quotes,positioning,calc,arrows.meta}
\usepackage[all]{xy}

%-----------------------Macros-------------------------------
\newcommand{\Wt}{\textup{Wt}}

%-------------------------Text Starts----------------------


\begin{document}

\maketitle

The binary dihedral group $\text{BD}_{4n}$ is a double cover of the standard dihedral group. It may be presented as a subgroup of $\SL(2,\bC)$ as follows:
\[
\text{BD}_{4n}:= \left\langle \sigma=\left(\begin{array}{cc} \varepsilon & 0 \\ 0 & \varepsilon^{-1} \end{array} \right), \tau=\left( \begin{array}{cc} 0 & 1 \\ -1 & 0 \end{array} \right) \right\rangle
\]
with $\varepsilon$ a primitive $2n$th root of unity.
The group BD$_{4n}$ corresponds to the Dynkin diagram $D_{n+2}$ in the classification of subgroups of $\SL(2,\bC)$.
It has $n+3$ irreducible representations: four of rank 1 and the rest are of rank 2.
We will label the four rank 1 irreducibles by $\rho_0,\rho_1,\rho_{n+1},\rho_{n+2}$ and the rank 2 irreducibles will be numbered $\rho_2, \ldots, \rho_{n}$.
The vector space underlying $\rho_i$ will be denoted $V_i$.
We fix an $n$ and from now own refer to BD$_{4n}$ by $\Gamma$.

We take $G:=\oplus_i \GL(V_i)$, this comes with a natural functor $\Rep(G) \rightarrow \Rep(\Gamma)$ while both categories come with forgetful functors to $\Vect(\bfk)$.
We seek to define a strong symmetric monoidal lift $F \colon \Rep(G) \rightarrow \Vect(\bfk)$.
This coincides with notion defined in Definition~4.1 of the manuscript by the comment at the end of Section 4 of the manuscript.
On the level of objects and morphisms this is almost tautological.
It remains to fill in the monoidal data.
We will do this as is set out by the manuscript: first we express $\Rep(\Gamma)$ as a {\em quotient} or {\em extended category} of $\Rep(G)$.

To do so, it helps to use the McKay quiver of $\Gamma$ to encode some aspects of the monoidal structure.
Recall that the McKay quiver of $\Gamma$ has a vertex for each isomorphism class of irreducible representation with $\dim(\Hom(\rho_i,\rho_j \otimes \rho_2))$ arrows from a vertex $i$ to a vertex $j$.
This implies that the number of paths of length $n$ between $i$ and $j$ is the same as the dimension of $\Hom(\rho_i, \rho_j \otimes \rho_2^{\otimes n})$.
We get relations on the McKay quiver by setting paths of length two corresponding to homomorphisms factoring through $\rho_j \otimes \wedge^2 \rho_2$ to zero; these are called the {\em preprojective relations}.

We let $d_i$ denote the dimension of $V_i$ and use $L_i:= \wedge^{d_i} \,V_i$ to denote the determinant representation of the group $\GL(V_i)$.
The following are going to be our set of {\em indeterminant} isomorphisms:
\begin{align}
    &\bfk \rightarrow V_0 \\
%\bfx \colon &T_0 \longrightarrow V \label{fd:x} \\
\alpha_i \colon &L_i \otimes L_i \longrightarrow \bigotimes_{i \rightarrow j} L_j \quad \text{for all} \,\, i\neq 0 \label{fd:alpha} \\
 P_i \colon &V_i \otimes V_2 \longrightarrow \bigoplus_{i \rightarrow j} V_j \quad \text{for all} \,\, i\neq 0. \label{fd:P}
\end{align}
From this we get a morphism of particular importance $\alpha_0:= \otimes_i \,\alpha_i^{d_i}: L_2 \rightarrow L_0$.
{\red For what follows, we are use duals and top wedges but I am pretty sure this can be phrased with out the need for a rigid monoidal or $\lambda$-ring structure.}
We write $P_i^j$ for the projection of $P_i$ to the $j$-th summand. 
Furthermore, we write $$ P_{i^\vee}: = \bigoplus_{j \rightarrow i} (P_j^i)^T \colon V_i^\vee \otimes V_2 \longrightarrow \bigoplus_{i \rightarrow j} V_j^\vee.$$

We then quotient out by the following set of {\em relations}:
\begin{align}
    \wedge^{d_i+r} P_i &= \alpha_0^r \circ \alpha_i \circ \wedge^{d_i-r} P_{i^\vee}  \label{eq1:cons2} \\
    \alpha_0^r \circ \wedge^{d_i-r} P_i &= \alpha_i \circ \wedge^{d_i+r} P_{i^\vee}  \label{eq2:cons2} 
\end{align}
here $r =0,\ldots, d_i$ and all $i$.

We unpack Equations~(\ref{eq1:cons2}) and (\ref{eq2:cons2}) to show they are reasonable: we have natural isomorphisms
$$I^\dagger_r(W) \colon \wedge^{d+r} W \rightarrow \wedge^{d-r} W^\vee \otimes L_W$$ for $W$ a $d$-dimensional vector space and $r= -d, \ldots, d$.
Precomposing $\wedge^{d_i+r} P_i$ by the inverse of isomorphism $I^{\dagger}_r(V_i \otimes V_2)$ and postcomposing $I^{\dagger}_r(\oplus V_j)$ gives a field that differs from $\alpha_i \circ P_{i^\vee}$ by some multiple of $L_2$.
The factor $\alpha_0^r$ is used to fix this discrepancy.

We note, {\red without proof for now}, that these relations imply the commutativity of the following diagram:
\begin{equation*}
    \xymatrix{V_i \otimes V_2 \otimes V_2 \ar[r]^-{P_i \otimes \text{id}} \ar[d]^{\text{id}} & \big(\bigoplus_{i \rightarrow j}  V_j\big) \otimes V_2 \ar[r]^-{\oplus_{i \rightarrow j} P_j} & \bigoplus_{i \rightarrow j \rightarrow k} V_k \ar[d]^{\text{id}} \\
    V_i \otimes V_2 \otimes V_2 \ar[r]^-{\text{id} \otimes \text{det}} & V_i \otimes L_2 \ar[r]^-{\alpha_0 \cdot \text{Preproj}} & \bigoplus_{i \rightarrow j \rightarrow k} V_k.}
\end{equation*}

\begin{proposition}
The quotient of the category $\Rep(G)$ by the indeterminants and relations above is equivalent to $\Rep(\Gamma)$ as a symmetric monoidal category.
\end{proposition}



\end{document}
